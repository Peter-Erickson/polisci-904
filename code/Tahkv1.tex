\PassOptionsToPackage{unicode=true}{hyperref} % options for packages loaded elsewhere
\PassOptionsToPackage{hyphens}{url}
\PassOptionsToPackage{dvipsnames,svgnames*,x11names*}{xcolor}
%
\documentclass[12pt,]{article}
\usepackage{lmodern}
\usepackage{amssymb,amsmath}
\usepackage{ifxetex,ifluatex}
\usepackage{fixltx2e} % provides \textsubscript
\ifnum 0\ifxetex 1\fi\ifluatex 1\fi=0 % if pdftex
  \usepackage[T1]{fontenc}
  \usepackage[utf8]{inputenc}
  \usepackage{textcomp} % provides euro and other symbols
\else % if luatex or xelatex
  \usepackage{unicode-math}
  \defaultfontfeatures{Ligatures=TeX,Scale=MatchLowercase}
\fi
% use upquote if available, for straight quotes in verbatim environments
\IfFileExists{upquote.sty}{\usepackage{upquote}}{}
% use microtype if available
\IfFileExists{microtype.sty}{%
\usepackage[]{microtype}
\UseMicrotypeSet[protrusion]{basicmath} % disable protrusion for tt fonts
}{}
\usepackage{xcolor}
\usepackage{hyperref}
\hypersetup{
            pdftitle={When the Pen Becomes a Sword: Retired US Military Officers and Op-Ed Use in Contemporary America},
            pdfauthor={Peter M. Erickson},
            colorlinks=true,
            linkcolor=magenta,
            filecolor=Maroon,
            citecolor=black,
            urlcolor=blue,
            breaklinks=true}
\urlstyle{same}  % don't use monospace font for urls
\usepackage[margin = 1.15in]{geometry}
\usepackage{longtable,booktabs}
% Fix footnotes in tables (requires footnote package)
\IfFileExists{footnote.sty}{\usepackage{footnote}\makesavenoteenv{longtable}}{}
\usepackage{graphicx,grffile}
\makeatletter
\def\maxwidth{\ifdim\Gin@nat@width>\linewidth\linewidth\else\Gin@nat@width\fi}
\def\maxheight{\ifdim\Gin@nat@height>\textheight\textheight\else\Gin@nat@height\fi}
\makeatother
% Scale images if necessary, so that they will not overflow the page
% margins by default, and it is still possible to overwrite the defaults
% using explicit options in \includegraphics[width, height, ...]{}
\setkeys{Gin}{width=\maxwidth,height=\maxheight,keepaspectratio}
\setlength{\emergencystretch}{3em}  % prevent overfull lines
\providecommand{\tightlist}{%
  \setlength{\itemsep}{0pt}\setlength{\parskip}{0pt}}
\setcounter{secnumdepth}{0}
% Redefines (sub)paragraphs to behave more like sections
\ifx\paragraph\undefined\else
\let\oldparagraph\paragraph
\renewcommand{\paragraph}[1]{\oldparagraph{#1}\mbox{}}
\fi
\ifx\subparagraph\undefined\else
\let\oldsubparagraph\subparagraph
\renewcommand{\subparagraph}[1]{\oldsubparagraph{#1}\mbox{}}
\fi

% set default figure placement to htbp
\makeatletter
\def\fps@figure{htbp}
\makeatother

% there are a few ways to include latex options.
% 1: as a preamble file (like this one)
% 2: directly into the header-includes YAML variable (global)
% 3: output templates. For more info on templates,
%    see <https://bookdown.org/yihui/rmarkdown/rticles-templates.html>
%    and <https://bookdown.org/yihui/rmarkdown/rticles-usage.html>

% These packages control the fonts used in PDF output.
% The standard LaTeX engine (PDFLaTeX) has a limited selection of fonts.
% See <https://www.tug.org/FontCatalogue/>

\usepackage{libertine}                 % serif font
\usepackage{libertinust1math}          % matching math font
\usepackage[scaled = 0.95, varqu]{zi4} % teletype (monospace/code) font

% The XeLaTeX engine gives you more control to specify fonts
%   at the expense of having poorer typesetting principles
%   (but xelatex typesetting is getting better)

% this package enables decimal-aligned table columns
\usepackage{dcolumn}

% the default PDF template for R Markdown loads a lot of convenient packages
% so you don't actually have to mess with them all that much.
\usepackage{setspace}
\usepackage{multirow}
\usepackage{indentfirst}
\usepackage{caption}
\usepackage{booktabs}
\usepackage{longtable}
\usepackage{array}
\usepackage{multirow}
\usepackage{wrapfig}
\usepackage{float}
\usepackage{colortbl}
\usepackage{pdflscape}
\usepackage{tabu}
\usepackage{threeparttable}
\usepackage{threeparttablex}
\usepackage[normalem]{ulem}
\usepackage{makecell}
\usepackage{xcolor}
\usepackage[style=authoryear,]{biblatex}
\addbibresource{researchpaper-ps904.bib}

\title{When the Pen Becomes a Sword: Retired US Military Officers and Op-Ed Use in Contemporary America}
\author{Peter M. Erickson}
\date{December 03, 2020}

\begin{document}
\maketitle
\begin{abstract}
\emph{This chapter explores variation in the professionalism of opinion pieces authored by retired US military officers and published in major US newspapers since 1979. Drawing on an original data set almost 400 opinion-editorial pieces published by retired military officers in The Wall Street Journal, The Washington Post, The New York Times, The Los Angeles Times, and the USA Today since 1979, the paper argues there has been an increase in published opinion pieces which fail to uphold the concept of military professionalism as defined by historical military norms and standards. What explains this variation? The author argues that increases in political polarization and the prestige of the military account for the increase in the number of op-eds which challenge traditional conceptions and norms of military professionalism. The paper concludes by arguing that the concept of ``politicization of the military'' --- something that previous civil-military relations scholars have identified and described --- must be widened to account not only for efforts by politicians to use the military for political advantages, but also for the phenomenon of retired military officers engaging the public to achieve political ends. The author argues that such a development, if uncorrected, will cause significant damage to the military as a profession.}\\
\end{abstract}

\doublespacing

\captionsetup[table]{labelformat=empty}

\newpage

\hypertarget{introduction}{%
\section{Introduction}\label{introduction}}

In late May and early June 2020, racial tension and concerns about police brutality gripped the American public following the death of George Floyd, an unarmed black man, by police in Minneapolis. In some cities and communities across the nation, peaceful protests were followed by periods of rioting and looting. On June 1, after giving remarks at the Rose Garden, President Donald Trump walked to St.~John's Church after security forces visibly displaced throngs of protesters in neighboring Lafayette Square. Flanked by members of his cabinet, to include the Secretary of Defense, and with the Chairman of the Joint Chiefs of Staff not far behind, the President held up a Bible at the church, stared into the cameras of the media, and, several minutes later, returned to the White House. In the days and weeks that followed, a civil-military relations crisis ensued \autocite{feaver_military_2020}. The Secretary of Defense, Mark Esper, publicly disagreed with the President's threat to invoke the Insurrection Act of 1807 to use active-duty forces to quell domestic riots \autocite{esper_secretary_2020}, and General Mark Milley, the Chairman of the Joint Chiefs of Staff, later told senior military officers graduating from the National Defense University that he ``should not have been there'' (accompanying the President on the walk to St.~John's Church), as his presence ``created a perception of the military involved in domestic politics'' \autocite{milley_official_2020}.

Furthermore, several retired military officers weighed in not only on the Trump administration's handling of specific events near the White House, but also on the broader issues of race in America.\footnote{These officers included several retired four star generals and admirals}. However, there was marked variation in the tone and tenor of the remarks made by these officers: although each expressed deep concern about the prospect of active-duty forces confronting peaceful protesters, and thus, violating protections held by the First Amendment of the US Constitution \autocites{brooks_dismay_2020}{dempsey_former_2020}, some officers went much further in their criticisms, leveling direct insults of the President, including his fitness to lead \autocite{goldberg_james_2020}, his administration's broader policies on race in America \autocite{allen_moment_2020}, and his general leadership abilities \autocite{mullen_i_2020}. Moreover, the comments made by these retired senior leaders renewed interest, at least for a while, in the topic of retired military officer speaking out on current events - a topic that civil-military relations scholars have debated \autocites{kagan_let_2006}{brooks_let_2020}, but one that has yet, in the author's opinion, to be studied empirically and systematically. In a domestic political age in which polarization is high, the topic of retired military officer speech is important for a variety of reasons. Not only does the speech of retired military officers carry the potential to impact public opinion, but the very act of retired senior leaders speaking out may also compel those who are still in the military to follow suit. More importantly, the topics that military leaders choose to address in the public square, and the tone of their remarks, is reflective of the conceptualizations these officers have regarding their role in society and ultimately, their purpose and professional identity as military officers. Indeed, in 2020, retired as well as active duty military officers grabbed the pen and argued in support of a variety of topics that clearly pushed the traditional boundaries of professional conduct by military officers by openly arguing for a variety of topics related to President Donald Trump, including including race in America \autocite{askew_anti-racist_2020} and even removing the President from the White House \autocite{nagl__2020}.

This chapter examines variation in the publication of op-ed pieces by retired military officers since 1979. There are two main arguments of this chapter. The first argument is that the variation in the number, topic addressed, and tone of opinion-editorial publications published by retired military officers indicates a shift over the past 40 years in the way a number of retired military officers think about their normative roles as members of the military institution and how they interact with the public in America. The second argument is that the driver of the variation in the number, topic, and tone of op-eds authored by retired military officers is mainly the increase in political polarization that has occurred in America since the 1970s. While there are other factors that influence also responsible, such as the increase in institutional prestige of the military over the same time period, the empirical story reveals that political polarization holds more explanatory power than other factors in explaining the variation of retired military officer speech in America.

This paper proceeds in three parts. First, the paper connects the concept of military ``role beliefs'' to retired military officer speech. Second, the paper explores and analyzes an original data set of nearly 390 opinion-editorial pieces published in major US newspapers by retired military officers since the late 1970s. This analysis ultimately demonstrates that although the vast majority of op-eds published by retired military officers maintain professional standards, there is clear evidence that retired military officers are using the pen both to argue about topics that fall outside of the realm of traditional national security topics, and to violate the principle of civilian control of the military when doing so. The analysis also shows that these trends have worsened as the American political climate has become increasingly polarized over the past several decades. Third, thechapter concludes by exploring the implications of these results on the military as an institution, and specifically, on its ability to remain professional in a highly polarized setting that will likely endure.

\singlespace

\hypertarget{how-the-military-views-itself}{%
\section{How the Military Views Itself}\label{how-the-military-views-itself}}

\doublespace

\hypertarget{military-role-beliefs}{%
\subsection{Military Role Beliefs}\label{military-role-beliefs}}

The concept of military ``role beliefs'' was introduced by J. Samuel Fitch in his book, \emph{The Armed Forces and Democracy in Latin America.} In the book, Fitch describes military ``role beliefs'' as the perceptions held by military officers regarding their normative role in politics \autocite[61]{fitch_armed_1998}. Though Fitch's focus is on militaries in Latin America, it is possible to use the concept of ``role beliefs'' to develop a normative starting point applicable to military officers in the United States.

Two foundations for developing a starting point for military ``role beliefs'' in the United States are the tandem norms of civilian control of the military, and the idea of the military acting as a non-partisan institution. Though such norms have been episodically challenged throughout American history in a variety of ways and during specific instances, these norms have nonetheless been both foundational and aspirational, particularly in the post World War Two period \autocite{weigley_american_2001}. The norm of civilian control of the military stems directly from the US Constitution, which expressly grants control of the military (as well as specific responsibilities associated with maintaining control) to the President, Congress, and the individual states \autocite{noauthor_us_1787-2}. Unlike the norm of civilian control, the norm of the military functioning as a non-partisan institution is not expressly spelled out in the Constitution, but rather the product of the American military's collective experiences and traditions in fighting and prevailing in the nation's wars and conflicts.

Further, a number of scholars have underscored the importance of these two norms, and the relationship between these norms and the military's ability to prevail during times of war. In his influential book, \emph{The Soldier and the State: The Theory and Politics of Civil-Military Relations}, Samuel Huntington argues in favor of maintaining separate operating spheres between the political leaders of a democracy and the military as an institution in order to ensure civilian control of the military by rendering the military ``politically sterile and neutral,'' or non-partisan \autocite[84]{huntington_soldier_1957}. Huntington goes on to argue that only by such an arrangement, which he called ``objective control of the military,'' will the military be capable of defeating significant security threats while also remaining apolitical \autocite[83-84]{huntington_soldier_1957}.

Having established the foundational norms of civilian control of the military and the military operating as an apolitical institution, the next step is to extrapolate how these norms might impact the public speech of military officers. This requires balancing these norms with the fact that military officers are nonetheless servants of the Republic who have accumulated knowledge and experiences over the course of sometimes lengthy careers (in the case of retired military officers, at least 20 and in some cases, upwards of 40 or even more years of experience), and thus, in many cases are qualified to weigh in on circumstances and situations that impact the nation. For instance, in the aftermath of the ``Revolt of the Generals'' in 2006, in which several retired military officers publicly criticized and called for the removal of Defense Secretary Donald Rumsfeld, Frederick Kagan argued in support of retired military officers having the freedom to weigh in policy debates, particularly those involving an active armed conflict, stating, ``there is no danger to the republic in a handful of retired generals speaking their minds. There is great danger in making vital decisions about an on-going armed struggle without hearing the views of all available experts'' \autocite{kagan_let_2006}. Kagan's remarks lend support to the notion that in democracies in particular, transparent and flowing information is a precious resource that helps leaders make decisions and, more importantly, the public to gain awareness of the state's actions {[}need cite{]}.

Nonetheless, there have been both formal and informal boundaries placed on various forms of political activities by military officers. Formally speaking, Department of Defense (DoD) Directive 1344.10 contains a ``standing list of political behavior do's and don'ts for members of the active duty military'' \autocite[1]{urben_wearing_2014}. This list includes authorizing members of the military to engage in activities such as voting and joining a partisan or non-partisan club and to attend this club's meetings when not in uniform, and prohibits members of the military from activities such as participating in partisan fundraising, or advocating for or against a partisan cause or figure on radio or television \autocite{department_of_defense_political_2008}. Informally, a number of recent Chairmen of the Joint Chiefs of Staff have likewise urged members of the military to refrain from engaging in overly political activities, particularly in election years, when such activity is more likely to negatively reflect on the military. For example, prior to the 2016 Presidential Election, then-Chairman of the Joint Chiefs of Staff General Joseph Dunford reminded the military that the basis upon which it is able to maintain the public's trust is it's ability to remain ``apolitical and neutral'' \autocite{dunford_upholding_2016}.

Based on these formal and informal prohibitions on the political activities conducted by military officers, it is possible to conclude that visible behaviors that are overtly partisan are to be avoided at all costs, and are inappropriate for military officers to engage in. It is also possible to extrapolate that the norm of civilian control of the military results in a corresponding prohibition on members of the military criticizing political leaders to the point that civilian control of the military is jeopardized or fails to be upheld. Where this line should be drawn in practice is not clear, however, because as Kagan has pointed out, criticisms can and often do point out blind spots in a plan or otherwise reveal shortcomings that may not otherwise have been identified. However, there is a point at which a criticism levied at a political leader by a military officer goes beyond criticism. Finally, it is possible to conclude that activities which build on the appropriate expertise and knowledge of the military officer are appropriate, provided that these activities do not violate the stipulations previously mentioned, i.e.~a non-partisan military, and civilian control of the military.

\hypertarget{opinion-editorial-publication-as-a-political-activity}{%
\subsection{Opinion-Editorial Publication as a Political Activity?}\label{opinion-editorial-publication-as-a-political-activity}}

We can now focus on the actual behavior of the military authoring op-ed pieces. Should these considered a form of political activity? I argue that it depends on what message the op-ed actually sends or is intended to send. An op-ed that strictly points out the pitfalls of the nation's combat strategy written by an experienced military officer may not be inappropriate, whereas an op-ed written by a retired military officer that argues in favor of a specific Presidential candidate is, at least by the standards put forth in this paper, entirely inappropriate. Therefore, at least two questions should be asked when examining an opinion piece authored by a retired military officer: first, does the piece violate either the norm of civilian control of the military or the military as an apolitical institution, and second, does the op-ed address an appropriate topic that a military officer is qualified to address?

Beyond these two questions, however, opinion-editorials are important because they serve as an informative barometer of how the military institution sees itself, and particularly, its role with respect to the American public. Members of the active military do not frequently write op-eds, but there have been notable exceptions over the years, perhaps none so notable as several written by then-Chairman of the Joint Chiefs of Staff General Colin Powell in the early 1990s. For instance, Powell presented the public with his view regarding the potential use of American military force in several developing ``hot spots'' across the world, including the Balkans, Somalia, and other places, during an October 8, 1992 piece entitled, ``Why Generals Get Nervous.'' Published in the \emph{New York Times}, Powell wrote in part, ``decisive means and results are always to be preferred, even if they are not always possible. So you bet I get nervous when so-called experts suggest that all we need is a little surgical bombing or a limited attack. When the desired result isn't obtained, a new set of experts then comes forward with talk of a little esclalation. History has not been kind to this approach'' \autocite{powell_opinion_1992}.

Powell clearly felt compelled to inform the public of how he, as the military's top officer, viewed the potential use of American military force abroad during the prevailing international context at the time. What is less known, however, is that Powell was also writing against a domestic backdrop in which there was an election coming up in less than a month that pitted the incumbent, George W. Bush, against Democratic challenger Bill Clinton. The two candidates differed with respect to foreign policy on a number of issues, and Bill Clinton was not highly respected by the military establishment, a fact that would impact the military's relationship with Clinton during his early years in office {[}need cite{]}. Whether it was proper for Powell to write such an op-ed truly depends: one could argue that the nation's top general was simply informing the public based on his years of individual experience serving in the military, in which case, from a democratic transparency standpoint, the op-ed well served the Republic; alternately, one could also argue that by writing during an election year, Powell's remarks brazenly staked the claims of the military ahead of a potential Presidential transition (which turned out to be the case), and in the process, limited the President's options for potentially using force abroad during his early days in office. The first explanation argues that from the standpoint of civil-military relations, there was no problem with Powell's opinion-editorial, while the second argues that the op-ed was nothing but problematic.

\hypertarget{retired-military-officers}{%
\subsection{Retired Military Officers}\label{retired-military-officers}}

One might argue that the logic employed thus far is seemingly plausible for members of the active military, but that the cautionary tone and reservations regarding certain types of political activity should not apply for members of the retired military community. And indeed, there are obvious distinctions between those who are on active duty and those who have retired from the service. These distinctions are made clear in fact in portions of DoD 1344.10 \autocite{department_of_defense_political_2008}. But the fact of the matter is that, especially for retired military officers who achieved general or flag officer rank, these retired officers sometimes carry outsized influence and impact when they engage in activities that approach on the political. During the Presidential Campaign of 2016, Retired Army General Martin Dempsey published an opinion-editorial piece in \emph{The Washington Post} which criticized retired Army Major General Mike Flynn and Marine Corps General John Allen for participating in the respective Republican and Democratic conventions. Dempsey wrote, ``As generals, they have an obligation to uphold our apolitical traditions. They have just made the task of their successors -- who continue to serve in uniform and are accountable for our security - more complicated. It was a mistake for them to participate as they did. It was a mistake for our presidential candidates to ask them to do so'' \autocite{dempsey_military_2016}.

Of course, speaking at a political convention on behalf of a Presidential candidate is, if not formally, at least informally the textbook definition of a political activity. Not every opinion-editorial piece supports a political candidate's agenda, or disparages the political views of another. Thus, it is important to remember that although an op-ed can be written in such a way that it is a politically inappropriate form of political behavior, it need not be. It is precisely on this point that the norms of civilian control of the military, the military as an apolitical institution, and retired members of the military expressing views that are in accord with the knowledge and skills gained during their time in uniform are appropriate.

Moreover, because multistar generals and admirals in the contemporary military possess in many cases upwards of 30 and sometimes 40 years of service, it is entirely reasonable to conclude that these high-ranking officers in particular have been privy to decisions and experiences occurring at or near the highest levels of government. Having interacted with Presidents, members of the cabinet, members of Congress, and foreign dignitaries, the experiences of retired general and flag officers are not limited to only tactical matters. Though they possess an abundance of expertise in many tactical matters that pertain to warfighting, they are also familiar with issues ranging from how wars should be fought, to future security threats looming on the horizon, to fixing the institutional culture of large organizations such as the military services, and much more.

Nonetheless, they are not experts in everything, nor should they try to be. Even General of the Army Douglas MacArthur, one of several men who ever wore the five star rank, and a hero of World War Two who later clashed with President Truman regarding American strategic goals in Asia, recognized that the military had focus areas, even if he did not personally exhibit these limitations while on active duty. Addressing the Corps of Cadets at West Point in 1962, the then-82 year old retired General told the Corps:

\singlespacing

\begin{quote}
\textbf{Yours is the profession of arms, the will to win, the sure knowledge that in war there is no substitute for victory, that if you lose, the Nation will be destroyed, that the very obsession of your public service must be Duty, Honor, Country. Others will debate the controversial issues, national and international, which divide men's minds. But serene, calm, aloof, you stand as the Nation's war guardians, as its lifeguards from the raging tides of international conflict, as its gladiators in the arena of battle. For a century and a half you have defended, guarded and protected its hallowed traditions of liberty and freedom, of right and justice. Let civilian voices argue the merits or demerits of our processes of government. Whether our strength is being sapped by deficit financing indulged in too long, by federal paternalism grown too mighty, by power groups grown too arrogant, by politics grown too corrupt, by crime grown too rampant, by morals grown too low, by taxes grown too high, by extremists grown too violent; whether our personal liberties are as firm and complete as they should be. These great national problems are not for your professional participation or military solution. Your guidepost stands out like a tenfold beacon in the night: Duty, Honor, Country} \autocite{macarthur_duty_1962}.
\end{quote}

\doublespacing

By systematically examining the op-eds written by retired military officers in major US publications over time, it is possible, I contend, to draw inferences about how the concept of a ``role belief'' - conceptions of one's normative role in politics - has changed among retired US military officers (if at all) \autocite{fitch_armed_1998}. Op-eds are an interesting and appropriate medium to examine if differentiation in ``role beliefs'' is an object of interest. The first reason why this is the case is because as previously stated, the act of crafting and publishing an op-ed does not necessarily constitute a political activity, though depending on content, they can be. A second and releated reason is because unlike other forms of engaging with the press, such as giving interviews on television, crafting an op-ed requires the author to deliberately prepare and submit a viewpoint for publication. In major newspapers, this is often a competitive process. It is not uncommon to see opinion pieces in prominent newspapers published by heads of state, members of congress, governmental figures, and leaders of industry and academia. A deliberate and carefully-crafted message further underscores the degree to which the author believes the message he or she is sending. It is difficult to ``misspeak'' during or to be ``misquoted'' from an opinion-editorial. The author generally agrees with the message he or she is sending.

When examining op-eds authored by retired military officers over time to detect potential shifts in normative role beliefs, the key area of focus must be the degree to which retired military officers have upheld vice strayed from the norms of civilian control and non-partisanship of the military, as well as the degree to which retired officers have addressed topics that are appropriately rooted in relevant military expertise. If there are patterns of violating these norms that can be detected, it may indicate that the role beliefs of certain members of the retired military community have likewise changed. On the other hand, if there has been little deviation over time from these foundational norms, it is possible to conclude that military retirees view their normative roles in politics (role beliefs) in similar ways as previous generations of retired military officers.

\hypertarget{hypotheses}{%
\subsection{Hypotheses}\label{hypotheses}}

The hypotheses I wish to examine, then, are as follows:

\emph{H1: When political polarization is high, the military (or certain military figures) will increasingly criticize civilian leadership (indicating a drop in the principle of civilian control).}

\emph{H2: When political polarization is low, the military (or certain military figures) will publicly weigh in on traditional national security debates (and not on domestic political concerns or issues outside of relevant military expertise).}

\emph{H3: During periods of high prestige and credibility, the frequency of military involvement in quasi-political activities (such as writing op-eds) will increase relative to periods of low prestige and credibility.}

\emph{H4: When both political polarization and high prestige are present, the military will be increasingly vocal and at higher frequency about certain domestic ideological or partisan causes than when these factors are not present.}

\emph{H5: During periods of active ground combat, military elites will likely comment on the war, but with less frequency on other domestic issues.}

Some caution must be taken up front so as to not overpromise and under-deliver concerning what the data is actually capable of presenting. The first caution to consider is that opinion-editorial publications by nature reflect the demand of the press to publish the viewpoints of retired military officers. Moreover, because op-eds are generally provocative and at least to some degree designed to advocate for or against a certain course of action, it is possible that op-eds may generally present somewhat contentious material. Additionally, this means that under periods of harmonious civil-military relations, there is either less of a demand from the press to publish the viewpoints of retired military officers, or less of a supply side effect for retired military officers to write about topics in the first place. Certain exogenous events, such as wars, will impact whether retired military officers speak out. A second cautionary note is that over the course of several decades, underlying societal changes - such as political polarization - have occurred that likely impact either or both the willingess of the press to publish the viewpoints of military officers, or the desire of military officers to write about certain topics in the first place. It is important that these societal-level effects be identified and, if possible, controlled for.

\hypertarget{methodological-approach-and-the-data}{%
\section{Methodological Approach and the Data}\label{methodological-approach-and-the-data}}

The methodological approach used in this analysis consists of two steps. The first is to analyze the topics which retired military officers have been writing about, and the second is to determine, regardless of the topic addressed in a publication, whether the op-ed has violated norms of non-partisanship or civilian control of the military. The data consists of opinion-editorial and letters to the editor publications authored by retired military officers and published in major US newspapers from 1979-2020. These newspapers consist of \emph{The Wall Street Journal} (WSJ), \emph{The New York Times} (NYT), \emph{The Washington Post} (WaPo), \emph{The Los Angeles Times}, and \emph{The USA Today}.

I include both opinion-editorials and letters to the editor because both types of publications display a viewpoint that required a deliberate effort by the officer to craft and submit to the particular newspaper. I do not include published interviews with retired military officers, nor reprinted editorials that were first published before 1979, such as those reprinted for the benefit of readers sometime after 1979. To assemble the data, I use a series of key-word searches, such as ``retired military officer,'' ``retired from the Army, Navy, Marine Corps, Air Force,'' ``former Chairman of the Joint Chiefs of Staff,'' ``General,'' ``Admiral,'' and so forth. I conduct these key word searches in several different databases, including Factiva, Lexus-Uni, and ProQuest Historical Newspapers. \footnote{For exact replication of the database search parameters, please see the Appendix.} I include in my search publications that were made available both in print and online format, though I did not double-count: if the same op-ed was published both in print and online and published by the same author, I count the article only one time.

There are both advantages and disadvantages to confining my data to select US newspapers, as I have. The primary advantage is that these sources offer longevity across a time period of nearly forty-two years; in doing so, they provide the researcher with an opportunity to detect patterns across several decades. The primary disadvantage is that by limiting myself to only the five newspapers I examine, I am not including a large number of modern social media and blogging sites, contemporary webpages, and other digital offerings that showcase the opinions of retired military officers.

Although social media has taken over as the primary means in which a majority of Americans receive their news today \autocite{martin_how_2018}, the methodological approach adopted in this analysis allows for an examination of the same sources over time. More importantly, even after accounting for the advent of online editions of newspapers in the late 1990s and early 2000s, the overall circulation of newspapers (print and online editions combined) is still down today compared to earlier years. As a means of rough comparison, estimated total daily newspaper circulation in the US in 2000 was 62 million for weekday and 54 million for weekend editions, respectively, while in 2018, the same figures were 29 million and 31 million \autocite{barthel_leading_2019}.

In some instances, especially where the same retired military officers authored multiple publications, the officer did not always refer to him or herself as a retired military officer. This occurred in several instances in which the officer was, ostensibly, writing from the perspective of someone other than a retired military officer, such as that of a political appointee to a high level governmental position (after retirement from the military, Army General Barry McCaffrey, for instance, served as President Clinton's Director of the Office of National Drug Control Policy). Nonethless, because these officers had previously identified themselves as retired military officers in earlier publications, and thus, because members of the public might reasonably recognize the author as a prominent retired military figure, I include the subsequent publications in the data, and simply control for whether the author identified him or herself as a retired military officer in a specific publication.

\hypertarget{summary-statistics}{%
\subsection{Summary Statistics}\label{summary-statistics}}

The data set includes a total of 390 observations, penned by 216 distinct different authors or author teams. Eight of the observations are written by anonymous authors, but it is unknown whether the same unknown author penned several of these eight observations. Included in the data are cases in which at least one retired military officer co-authored a piece with one or more civilian authors. \footnote{There are several co-authored pieces. For example, where General (Retired) David Petraeus wrote one publication and another one was written by General (Retired) David Petraeus and Michael O'Hanlon, I counted these as two different authors for coding purposes.} There are also several retired military officers who penned several observations. The most prolific of these officers includes Retired Army Lieutenant General William Odom (22 publications in total), Retired Air Force General Michael Hayden (20 publications in total), and Retired Marine Lieutenant Colonel and Former National Security Advisor Robert McFarlane (18 publications).

Figure One, below, plots the yearly count of publications authored by retired military officers across all of the major newspapers. The smoothed trendline indicates that the average yearly count of publications rose steadily from 1979 until the early 2000s, at which time, a decline occurred until approximately 2008. From 2008 until the present day, the average yearly count of publications authored by retired military officers in the major US newspapers examined in this study rose heavily during the Obama administration years before briefly falling, but then resuming the high count during the Trump administration.

It is noteworthy, from a macro perspective, to consider all of the impacts on the US military over the past four decades and more. Numerous substantive events, debates, and changes have occurred in this time period for retired military officers to weigh in on. In terms of external threats and wars, this period includes the final years of the Cold War against the Soviet Union (through 1991), the Gulf War in Iraq (1990-1991), operations in the Balkans (primarily 1993-1995), and the post-9/11 wars in Iraq (2003-present day) and Afghanistan (2001-present day), as well as a host of smaller but important military operations in Grenada (1983), Panama (1989), Somalia (1992-1993), and Haiti (1994-1995). The military has also undergone significant social and institutional changes in the time period examined. In the late 1970s, the military had ended the draft only years before (1973) following the conclusion of the Vietnam War. The debates over ``Don't Ask, Don't Tell'' (instituted in 1994 and repealed in 2011), as well as policies regarding women in the combat arms (2015) are also included in the data, as are numerous annual discussions regarding the size of military budgets.

\begin{verbatim}
## `geom_smooth()` using formula 'y ~ x'
\end{verbatim}

\begin{center}\includegraphics{Tahkv1_files/figure-latex/figure one-1} \end{center}

In addition to collecting the yearly count of publications, biographical information of the author is collected, including the service in which the author served, the rank achieved by the author during his or her time in service, and the sex of the author. Summary statistics for these variables are included in Table 1, below.

Insert table of yearly count, rank, and sex, as well as a panel facet graph of Figure 1, above, by source.

Describe the other variables and coding scheme.

A bar graph by decade across all sources collectively of \# of publication by topic count. Or do this later.

Of note, when assembling the data, some authors are explicitly identified in some of their publications as retired military officers, but not in others.\footnote{This was true of several individuals, including Alexander Haig, a retired Army General who later served as Secretary of State under Ronald Reagan; Robert McFarlane, a retired Marine Corps Lieutenant Colonel who later served as the National Security Adviser to President Reagan, Andrew Bacevich, a retired Army Colonel who then became a prominent historian and author; and Michael Hayden, a retired Air Force General who served as director of the National Security Agency and the Central Intelligence Agency}. I conduct a separate search for all opinion pieces penned by these authors, regardless of whether they were identified as a retired military officer or not, concluding that it is best do so for coding purposes.

Attempting to reconcile the fact that the military as an institution prides itself on being a non-partisan and apolitical institution, that newspaper circulations (even after accounting for digital offerings) are substantially down over the past 40 years, and that the US political environment is increasingly polarized, I adopt the two following hypotheses:

\emph{\(H_1\): The majority of op-eds published by retired military officers over the time period examine will be apolitical and non-partisan in tone.}

\emph{\(H_2\): Given that polarization has become increasingly pronounced in the United States over the time period examined, there will be an increase in recent years in the number of op-eds written about topics that are not relevant to military expertise, or that reveal inappropriate partisan sentiments. Such a trend may indicate that increasing political polarization is impacting military professionalism among a small but important part of the military institution, retired military officers.}

\hypertarget{first-layer-analysis-topics-addressed}{%
\section{First Layer Analysis: Topics Addressed}\label{first-layer-analysis-topics-addressed}}

In the first level of analysis, I code each observation into one of several binary categories that corresponds to the overall topic or subject of the publication. The goal is to create a coding scheme that captures a concise number of topical descriptions that are still sufficiently broad enough to reveal variation in the topic addressed by retired military authors. I develop a total of six topical variables, and each observation takes on the value of 1 if the publication corresponds to the general definition of the variable, and 0 otherwise.

These topical variables are: war fighting and operations, if the piece advocates for or denounces a particular course of action directly related to operational or war fighting decisions of a clear military nature; troop support and institutional culture, if the publication generally highlights the work and sacrifice of members of the armed forces, defends the work of specific members of the Armed Forces, or makes a call for action to improve a service's institutional culture; strategic advice, if the piece is generally written to offer some sort of strategic advice for the nation, to include the wisdom of pursuing particular alliances or agreements, or adopting a specific weapons platform or military budget; military social policy, if the piece advocates for or denounces a particular social policy that \emph{directly influences} the armed forces (Examples include: debates over keeping or removing the ``Don't Ask, Don't Tell'' policy, opening up combat roles to women, the role of Transgender members of the armed forces); domestic policy, if the piece advocates for or denounces a position on a policy that \emph{does not directly} influence the armed forces in its day to day activities (Examples include: support for the second amendment, national immigration policy, and whether confederate statues should be taken down or left up); and political figures, if the piece advocates for or denounces a particular position held by a political administration, or \emph{if the main point of the piece} is to explicitly defend or attack a political figure by name (Examples: endorsing a Presidential candidate for office, or for attacking a President or member of Congress for a decision that was made).

Up front, separating opinion-editorial publications into one of several distinct categories is an imperfect process. It requires a thoroughly objective reading of the publication as possible. To aid the coding process, I identify the thesis statement of each opinion piece, which in most cases is identifiable in either the lead or concluding paragraph of the opinion piece. In most cases, identifying and re-reading the thesis of each op-ed makes it possible to reasonably classify the publication by topic. Two additional points, however, are worth emphasizing here.

First, opinion pieces are intended to argue for something, and thus are, to an extent, designed to be at least mildly provocative. With this in mind, I did not necessarily code pieces that called for a political figure or institution to take a specific action into the category of \emph{figures by name}. Again, the identification of the publication's thesis statement aided significantly in identifying the main thrust of the argument, and thus, in classifying the argument by topic. Second, there were some instances in which I had to choose between placing a publication into one of two categories. However, in instances of choosing between two categories, the choice to classify between categories was usually between categories that were reasonably close. For example, there were a few instances of choosing to classify a publication between the war fighting and operations category and the strategic advice category, but no instances of choosing between the war fighting and operations category and the domestic policy category. Table 3 reveals the breakdown of all examined publications by article type.

\begin{table}

\caption{\label{tab:table3}Publications by Subject}
\centering
\fontsize{10}{12}\selectfont
\begin{tabu} to \linewidth {>{\raggedright}X>{\raggedleft}X}
\hline
\textbf{Topic} & \textbf{n}\\
\hline
\cellcolor{gray!6}{BdgtWpnsTrps} & \cellcolor{gray!6}{37}\\
\hline
CivMilBalance & 19\\
\hline
\cellcolor{gray!6}{DomesticPolicy} & \cellcolor{gray!6}{38}\\
\hline
ForeignPolicy & 112\\
\hline
\cellcolor{gray!6}{ServiceCulture} & \cellcolor{gray!6}{13}\\
\hline
SocialPolicy & 9\\
\hline
\cellcolor{gray!6}{Support} & \cellcolor{gray!6}{44}\\
\hline
Warfighting & 118\\
\hline
\end{tabu}
\end{table}

An overwhelming majority (293 of 344, or 85.2\%) of the total number of opinion pieces in the data set fall into the categories of war fighting and operations, troop support and institutional culture, and strategic advice. Moreover, I conclude that these observations pose no problem for norms of military professionalism, as they are written on topics that I contend are clearly related to one's relevant military experience. The remaining 51 publications (14.8\% of the observations) are coded as falling under the other three categories of topics: social policy directly related to the military, domestic policy, and political figures by name. These observations require a second and more-thorough examination to determine whether the content of these publications indicate a drop in the norm of military professionalism.

\singlespace

\hypertarget{the-second-layer---failing-to-uphold-professionalism}{%
\section{The Second Layer - Failing to Uphold Professionalism?}\label{the-second-layer---failing-to-uphold-professionalism}}

\doublespace

Conclusively determining whether an observation failed to uphold military professionalism is once again an imperfect and somewhat subjective evaluation. However, the work of academic theorists and decades of military norms make clear that at a minimum, military professionalism emphasizes behavior by military officers that is relevant to one's military expertise, as well as absent of partisan expression or preferences. Recall that Retired Army General Douglas MacArthur cautioned the cadets at West Point in 1962 about involving themselves in matters that do not require a military solution (in MacArthur's words, of ``politics grown to corrupt,'' ``morals grown too low,'' and ``whether our personal liberties are as firm and complete as they should be'') \autocite{macarthur_duty_1962}, and that more recently, in 2016, Retired Army General Martin Dempsey lamented the participation of retired generals in the political process by endorsing political candidates and appearing at the conventions of political parties:

\singlespace

\begin{quote}
\textbf{The American people should not wonder where their military leaders draw the line between military advice and political preference. And our nation's soldiers, sailors, airmen and Marines should not wonder about the political leanings and motivations of their leaders. As generals, they have an obligation to uphold our apolitical traditions. They have just made the task of their successors -- who continue to serve in uniform and are accountable for our security - more complicated} \autocite{dempsey_military_2016}.
\end{quote}

\doublespace

Relying on these two central ideals of military professionalism, relevant military expertise and non-partisan/apolitical expression, allows the researcher to ask two important questions: first, is the author addressing a topic related and applicable to the officer's military expertise, and second, is the author making an argument without clearly conveying or expressing political or partisan preferences? If the answer to either of these two questions is ``no,'' then the publication might violate the concept of military professionalism. I therefore develop two new variables, \emph{questionable professionalism} and \emph{justification}. The first variable is binary and codes as `1' the observations which may violate the concept of military professionalism, and the second variable briefly captures the basis for a possible violation, i.e.~whether it is a violation of topic or of expression of political sentiment. The results of this analysis on the remaining 51 observations are displayed in Table 4, below. Several observations are worth noting.

\singlespace
\begingroup\fontsize{10}{12}\selectfont

\begin{longtable}[t]{>{\raggedright\arraybackslash}p{9.5em}|>{\raggedright\arraybackslash}p{15em}|>{\raggedright\arraybackslash}p{2.5em}|>{\raggedleft\arraybackslash}p{2.5em}|>{\raggedright\arraybackslash}p{5em}|>{\raggedleft\arraybackslash}p{6em}|r|r}
\caption{\label{tab:table4}Op-Eds authored by Retired Military Officers that Strongly Criticized Civilian Leaders, or were Expressly Partisan, 1979-2020}\\
\hline
\textbf{Author} & \textbf{Title} & \textbf{Source} & \textbf{PubYr} & \textbf{Topic} & \textbf{Crit\_Insub} & \textbf{Endorse\_Partisan} & \textbf{Role}\\
\hline
\cellcolor{gray!6}{Yancey, William} & \cellcolor{gray!6}{General Bashing, Then and Now} & \cellcolor{gray!6}{WSJ} & \cellcolor{gray!6}{1989} & \cellcolor{gray!6}{CivMilBalance} & \cellcolor{gray!6}{1} & \cellcolor{gray!6}{0} & \cellcolor{gray!6}{0}\\
\hline
Beckwith, Charlie & Somalia's Needless Deaths & WSJ & 1993 & Warfighting & 1 & 0 & 0\\
\hline
\cellcolor{gray!6}{Odom, William} & \cellcolor{gray!6}{Invade, Don't Bomb} & \cellcolor{gray!6}{WSJ} & \cellcolor{gray!6}{1994} & \cellcolor{gray!6}{Warfighting} & \cellcolor{gray!6}{1} & \cellcolor{gray!6}{0} & \cellcolor{gray!6}{0}\\
\hline
McCain, John & In Bosnia, Another Mistake & WSJ & 1994 & Warfighting & 1 & 0 & 1\\
\hline
\cellcolor{gray!6}{Odom, William} & \cellcolor{gray!6}{One Year? In Bosnia?} & \cellcolor{gray!6}{NYT} & \cellcolor{gray!6}{1995} & \cellcolor{gray!6}{Warfighting} & \cellcolor{gray!6}{1} & \cellcolor{gray!6}{0} & \cellcolor{gray!6}{0}\\
\hline
Odom, William & Chechnya, Freedom, and the Voice of Yeltsin Past & WaPO & 1996 & ForeignPolicy & 1 & 0 & 0\\
\hline
\cellcolor{gray!6}{Mcdonough, James} & \cellcolor{gray!6}{Clinton's Contempt for U.S. Soldiers} & \cellcolor{gray!6}{WSJ} & \cellcolor{gray!6}{1998} & \cellcolor{gray!6}{Warfighting} & \cellcolor{gray!6}{1} & \cellcolor{gray!6}{0} & \cellcolor{gray!6}{0}\\
\hline
Peters, Ralph & It's Wonks vs. Warlords, and Guess Who Wins Again? & WSJ & 1998 & Warfighting & 1 & 0 & 0\\
\hline
\cellcolor{gray!6}{Peters, Ralph} & \cellcolor{gray!6}{A Question of Leadership} & \cellcolor{gray!6}{WSJ} & \cellcolor{gray!6}{1998} & \cellcolor{gray!6}{Warfighting} & \cellcolor{gray!6}{1} & \cellcolor{gray!6}{0} & \cellcolor{gray!6}{0}\\
\hline
Krulak, Charles & Don't Politicize the Joint Chiefs & WSJ & 2000 & CivMilBalance & 0 & 1 & 0\\
\hline
\cellcolor{gray!6}{Odom, William} & \cellcolor{gray!6}{Buchanan Has it Backwards on Globalization} & \cellcolor{gray!6}{WSJ} & \cellcolor{gray!6}{2000} & \cellcolor{gray!6}{ForeignPolicy} & \cellcolor{gray!6}{0} & \cellcolor{gray!6}{1} & \cellcolor{gray!6}{0}\\
\hline
Shalikashvili, John & Kerry Proceeds With Caution & USA Today & 2004 & BdgtWpnsTrps & 0 & 1 & 0\\
\hline
\cellcolor{gray!6}{Clark, Wesley} & \cellcolor{gray!6}{Medals of Honor} & \cellcolor{gray!6}{NYT} & \cellcolor{gray!6}{2004} & \cellcolor{gray!6}{DomesticPolicy} & \cellcolor{gray!6}{0} & \cellcolor{gray!6}{1} & \cellcolor{gray!6}{0}\\
\hline
Shalikashvili, John & Old Soldiers Don't Have to Fade Away & WSJ & 2004 & DomesticPolicy & 0 & 1 & 0\\
\hline
\cellcolor{gray!6}{Franks, Tommy} & \cellcolor{gray!6}{Right Leader, Right Time} & \cellcolor{gray!6}{WSJ} & \cellcolor{gray!6}{2004} & \cellcolor{gray!6}{DomesticPolicy} & \cellcolor{gray!6}{0} & \cellcolor{gray!6}{1} & \cellcolor{gray!6}{0}\\
\hline
Whitlow, William & The Price of Giving Bad Advice & WaPO & 2004 & Warfighting & 1 & 0 & 0\\
\hline
\cellcolor{gray!6}{Franks, Tommy} & \cellcolor{gray!6}{War of Words} & \cellcolor{gray!6}{NYT} & \cellcolor{gray!6}{2004} & \cellcolor{gray!6}{Warfighting} & \cellcolor{gray!6}{0} & \cellcolor{gray!6}{1} & \cellcolor{gray!6}{0}\\
\hline
Clark, Wesley & Before It's Too Late in Iraq & WaPO & 2005 & Warfighting & 1 & 0 & 0\\
\hline
\cellcolor{gray!6}{Clark, Wesley} & \cellcolor{gray!6}{The Next Iraq Offensive} & \cellcolor{gray!6}{NYT} & \cellcolor{gray!6}{2005} & \cellcolor{gray!6}{Warfighting} & \cellcolor{gray!6}{1} & \cellcolor{gray!6}{0} & \cellcolor{gray!6}{0}\\
\hline
Eaton, Paul & A Top-Down Review for the Pentagon & NYT & 2006 & CivMilBalance & 1 & 0 & 0\\
\hline
\cellcolor{gray!6}{Crosby, John and McInerney, Thomas and Moore, Burton and Vallely, Paul} & \cellcolor{gray!6}{In Defense of Donald Rumsfeld} & \cellcolor{gray!6}{WSJ} & \cellcolor{gray!6}{2006} & \cellcolor{gray!6}{Support} & \cellcolor{gray!6}{1} & \cellcolor{gray!6}{0} & \cellcolor{gray!6}{0}\\
\hline
Murtha, John & Confessions of a 'Defeatocrat' & WaPO & 2006 & Warfighting & 1 & 0 & 1\\
\hline
\cellcolor{gray!6}{Eaton, Paul} & \cellcolor{gray!6}{An Army of One Less} & \cellcolor{gray!6}{NYT} & \cellcolor{gray!6}{2006} & \cellcolor{gray!6}{Warfighting} & \cellcolor{gray!6}{1} & \cellcolor{gray!6}{0} & \cellcolor{gray!6}{0}\\
\hline
Scales, Robert & A War the Pentagon Doesn't Want & WaPO & 2013 & Warfighting & 1 & 0 & 0\\
\hline
\cellcolor{gray!6}{Hayden, Michael and Mukasey, Michael} & \cellcolor{gray!6}{NSA Reform that Only ISIS Could Love} & \cellcolor{gray!6}{WSJ} & \cellcolor{gray!6}{2014} & \cellcolor{gray!6}{Warfighting} & \cellcolor{gray!6}{1} & \cellcolor{gray!6}{0} & \cellcolor{gray!6}{0}\\
\hline
McSally, Martha & Saving a Plane that Saves Lives & NYT & 2015 & BdgtWpnsTrps & 1 & 0 & 1\\
\hline
\cellcolor{gray!6}{Scales, Robert} & \cellcolor{gray!6}{Our Army is Breaking} & \cellcolor{gray!6}{WaPO} & \cellcolor{gray!6}{2015} & \cellcolor{gray!6}{BdgtWpnsTrps} & \cellcolor{gray!6}{1} & \cellcolor{gray!6}{0} & \cellcolor{gray!6}{0}\\
\hline
Petraeus, David and O'Hanlon, Michael & Afghanistan After Obama & WaPO & 2015 & Warfighting & 1 & 0 & 0\\
\hline
\cellcolor{gray!6}{Hayden, Michael} & \cellcolor{gray!6}{Classified Briefings and Candidates} & \cellcolor{gray!6}{NYT} & \cellcolor{gray!6}{2016} & \cellcolor{gray!6}{CivMilBalance} & \cellcolor{gray!6}{1} & \cellcolor{gray!6}{0} & \cellcolor{gray!6}{0}\\
\hline
Davis, Morris & Stop Meddling in the Bergdahl Case & NYT & 2016 & CivMilBalance & 1 & 0 & 0\\
\hline
\cellcolor{gray!6}{Hayden, Michael} & \cellcolor{gray!6}{Russia's Useful Fool} & \cellcolor{gray!6}{WaPO} & \cellcolor{gray!6}{2016} & \cellcolor{gray!6}{DomesticPolicy} & \cellcolor{gray!6}{1} & \cellcolor{gray!6}{0} & \cellcolor{gray!6}{0}\\
\hline
McCain, John & We Have a Stake in Syria, Yet We Have Done Nothing & WaPO & 2016 & ForeignPolicy & 1 & 0 & 0\\
\hline
\cellcolor{gray!6}{Hayden, Michael} & \cellcolor{gray!6}{Trump's Most Important New Partner} & \cellcolor{gray!6}{WaPO} & \cellcolor{gray!6}{2016} & \cellcolor{gray!6}{Warfighting} & \cellcolor{gray!6}{1} & \cellcolor{gray!6}{0} & \cellcolor{gray!6}{0}\\
\hline
Hayden, Michael & A Damaging Disregard for Intel & WaPO & 2016 & Warfighting & 1 & 0 & 0\\
\hline
\cellcolor{gray!6}{Hayden, Michael} & \cellcolor{gray!6}{The Travel Ban Hurts American Spies - and America} & \cellcolor{gray!6}{WaPO} & \cellcolor{gray!6}{2017} & \cellcolor{gray!6}{DomesticPolicy} & \cellcolor{gray!6}{1} & \cellcolor{gray!6}{0} & \cellcolor{gray!6}{0}\\
\hline
Hayden, Michael & Trump Proves He's Russia's Useful Fool & WaPO & 2017 & DomesticPolicy & 1 & 0 & 0\\
\hline
\cellcolor{gray!6}{Mullen, Mike} & \cellcolor{gray!6}{The Refugees We Need} & \cellcolor{gray!6}{NYT} & \cellcolor{gray!6}{2017} & \cellcolor{gray!6}{DomesticPolicy} & \cellcolor{gray!6}{1} & \cellcolor{gray!6}{0} & \cellcolor{gray!6}{0}\\
\hline
Mullen, Mike & Bannon Has No Place on the NSC & NYT & 2017 & ForeignPolicy & 1 & 0 & 0\\
\hline
\cellcolor{gray!6}{Hayden, Michael} & \cellcolor{gray!6}{Donald Trump is Undermining Intelligence Gathering} & \cellcolor{gray!6}{NYT} & \cellcolor{gray!6}{2017} & \cellcolor{gray!6}{Warfighting} & \cellcolor{gray!6}{1} & \cellcolor{gray!6}{0} & \cellcolor{gray!6}{0}\\
\hline
Hayden, Michael & The End of Intelligence & NYT & 2018 & DomesticPolicy & 1 & 0 & 0\\
\hline
\cellcolor{gray!6}{McRaven, William} & \cellcolor{gray!6}{Take My Security Clearance, too, Mr. President} & \cellcolor{gray!6}{WaPO} & \cellcolor{gray!6}{2018} & \cellcolor{gray!6}{DomesticPolicy} & \cellcolor{gray!6}{1} & \cellcolor{gray!6}{0} & \cellcolor{gray!6}{0}\\
\hline
Wilkerson, Lawrence and Wilson III, Isaiah and Adams, Gordon & Trump's Border Stunt is a Profound Betrayal of Our Military & NYT & 2018 & Warfighting & 1 & 0 & 0\\
\hline
\cellcolor{gray!6}{Nagl, John} & \cellcolor{gray!6}{Retired Generals Warned Us About Rumsfeld. Now They're Warning Us About Trump} & \cellcolor{gray!6}{WaPO} & \cellcolor{gray!6}{2019} & \cellcolor{gray!6}{DomesticPolicy} & \cellcolor{gray!6}{1} & \cellcolor{gray!6}{0} & \cellcolor{gray!6}{0}\\
\hline
Allen, John and Victor, David & Despite What Trump Says, Climate Change Threatens Our National Security & NYT & 2019 & DomesticPolicy & 1 & 0 & 0\\
\hline
\cellcolor{gray!6}{McRaven, William} & \cellcolor{gray!6}{Our Republic is Under Attack From the President} & \cellcolor{gray!6}{NYT} & \cellcolor{gray!6}{2019} & \cellcolor{gray!6}{DomesticPolicy} & \cellcolor{gray!6}{1} & \cellcolor{gray!6}{1} & \cellcolor{gray!6}{0}\\
\hline
McRaven, William & If Good Men Can’t Speak the Truth, We Should Be Deeply Afraid & WaPO & 2020 & DomesticPolicy & 1 & 0 & 0\\
\hline
\cellcolor{gray!6}{VanLandingham, Rachel and Corn, Geoffrey} & \cellcolor{gray!6}{Military Brass Finally Speaks Up on Trump} & \cellcolor{gray!6}{USA Today} & \cellcolor{gray!6}{2020} & \cellcolor{gray!6}{DomesticPolicy} & \cellcolor{gray!6}{1} & \cellcolor{gray!6}{0} & \cellcolor{gray!6}{0}\\
\hline
Seidule, Ty & West Point and its Cadets are not Campaign Props & WaPO & 2020 & DomesticPolicy & 1 & 0 & 0\\
\hline
\cellcolor{gray!6}{Vindman, Alexander} & \cellcolor{gray!6}{Coming Forward Ended My Career. I Still Believe Doing What's Right Matters} & \cellcolor{gray!6}{WaPO} & \cellcolor{gray!6}{2020} & \cellcolor{gray!6}{DomesticPolicy} & \cellcolor{gray!6}{1} & \cellcolor{gray!6}{0} & \cellcolor{gray!6}{0}\\
\hline
McRaven, William & Trump is Working to Actively Undermine the Postal Service - and Every Major U.S. Institution & WaPO & 2020 & DomesticPolicy & 1 & 1 & 0\\
\hline
\cellcolor{gray!6}{Vindman, Alexander and Gans, John} & \cellcolor{gray!6}{Trump Has Sold off America's Credibility for His Personal Gain} & \cellcolor{gray!6}{NYT} & \cellcolor{gray!6}{2020} & \cellcolor{gray!6}{DomesticPolicy} & \cellcolor{gray!6}{1} & \cellcolor{gray!6}{0} & \cellcolor{gray!6}{0}\\
\hline
McRaven, William & Biden Will Make America Lead Again; We Need a President with Decency and a Sense of Respect & WSJ & 2020 & DomesticPolicy & 1 & 1 & 0\\
\hline
\cellcolor{gray!6}{Allen, Thad} & \cellcolor{gray!6}{Trump is Failing to Provide For the Common Defense} & \cellcolor{gray!6}{WaPO} & \cellcolor{gray!6}{2020} & \cellcolor{gray!6}{DomesticPolicy} & \cellcolor{gray!6}{1} & \cellcolor{gray!6}{0} & \cellcolor{gray!6}{0}\\
\hline
Hayden, Michael et. Al & A Dangerous Purge & WaPO & 2020 & Warfighting & 1 & 0 & 0\\
\hline
\cellcolor{gray!6}{89 Former Defense Officials} & \cellcolor{gray!6}{The Military Must Never Be Used to Violate Constitutional Rights} & \cellcolor{gray!6}{WaPO} & \cellcolor{gray!6}{2020} & \cellcolor{gray!6}{Warfighting} & \cellcolor{gray!6}{1} & \cellcolor{gray!6}{0} & \cellcolor{gray!6}{0}\\
\hline
\end{longtable}
\endgroup{}
\doublespace

First, 17 of the 49 publications are, upon a closer reading, determined to not reflect possible violations of military professionalism. This is worth showcasing in order to draw attention to the fact that there are instances of retired military officers authoring opinion pieces which address the subjects of social policy related to the military, domestic policy, and figures by name while simultaneously upholding standards of military professionalism. However, the data reveals that in practice, such instances are questionable. The practical implication is simple and straightforward: if military officers decide to write about topics other than war fighting and operations, troop support and institutional culture, or strategic advice, they need to carefully craft their argument so as to not violate norms of military professionalism.

Second, and resulting from the first, is that across the entire data set, instances of potential violations of military professionalism in the op-eds published by retired military officers in major US newspapers is a somewhat rare occurrence. Indeed, the entire count of potentially questionable publications with respect to military professionalism across the data set is just 32 of 344 total observations, or 9.3\% of the entire sample. Thus, ample support exists to confirm the first hypothesis, that the vast majority of opinion pieces authored by retired military officers uphold standards of military professionalism.

A third and interesting finding, however, is that the first publishing year in which an opinion piece penned by a retired officer potentially failed to uphold professional standards is 1998, nearly halfway across the sample in terms of publishing year. Why is this the case? We will deal more with the suspected culprits of this trend, rising political polarization and rising institutional credibility of the military, later in this paper. The point to make is that without question, retired military officers before 1998 had plenty of opportunities either to air partisan preferences and/or to address topics that may have similarly pushed the limits of military professionalism. In fact, many of these retired officers would have been veterans of the Vietnam and Korean Wars, and perhaps, like Maxwell Taylor, an author who is in the sample data, veterans of World War Two. Moreover, the data shows that these retirees wrote opinion pieces ( a total of 46 observations were written before the first observation in Table 4 raises a question about military professionalism), but that in doing so, they stayed clearly within the bounds of military professionalism.

Fourth, Table 4 reveals that several retired military officers wrote publications not from the perspective of a retired military officer, but from the perspective of other professional and political roles that these officers served in after retiring from the military. These include op-eds by: Retired Air Force Lieutenant General Claudius Watts, who wrote an op-ed regarding the advantages of all-male education while serving as the president of the Citadel; Retired Army General Barry McCaffrey, who penned an opinion piece while serving in the administration of Bill Clinton as the Director of the Office of National Drug Control Policy; and Retired Marine Corps Colonel John Murtha, who also served in the US House of Representatives. Each of these authors addressed topics that one should reasonably expect them to on the basis of the roles each was serving in. This would include, in the case of John Murtha, making obviously partisan agreements and thus, revealing political preferences.

Table 4 reveals, fifth, that several publications are deemed as possibly having violated military professionalism in that they address topics that may be considered outside of the realm of proper military expertise. It is worth briefly commenting on several of these examples. In his \emph{Washington Post} editorial, ``Home Should Not Be a War Zone,'' Retired Army General Stanley McChrystal addresses the tragic nature of mass shootings in America. He states in the article, ``\ldots{}as a combat veteran and proud American, I believe we need a national response to the gun violence that threatens so many of our communities'' \autocite{mcchrystal_home_2016}. He later, in 2018, pens an op-ed entitled ``Good Riddance'' in which he discusses the evolution of his views on Robert E. Lee and other Confederate soldiers and symbols \autocite{mcchyrstal_stanley_good_2018}. Retired Army Brigadier General Ty Seidule similarly, in 2020, advocates for the renaming of several Army installations \autocite{seidule_ty_what_2020}. Other examples of observations coded as potentially violating military professionalism based on the topic of the publication include articles penned by Retired Navy Admiral Mike Mullen (entitled ``The Refugees We Need'') and co-authors Retired Army Lieutenant General Mark Hertling and Retired Naval Admiral Robert Natter (entitled ``Cutting Refugee Admissions Will Have Severe Consequences for the US Military'').

These articles are certainly not unreasonable in that they, in the opinion of the author of this article, express reasonable thoughts and ideas to many of the nation's pressing problems, including gun violence, the appropriate legacy of Confederate symbols, and national immigration policy. To be clear, many Americans agree with the stance these officers take in the articles they have published. One can even make a strong, perhaps even a compelling argument, why these officers might be uniquely qualified to address aspects of some of these topics in that they possess expertise commanding large and powerful organizations charged to carry out violence on behalf of the country (related to the issue of gun control), that they may have experience serving and leading troops on bases named after Confederate generals (related to the issue of Confederate symbols and bases named after Confederate generals), and that they have have considerable experience serving alongside soldiers of multiple ethnicities as well as considerable experience in Iraq and Afghanistan (related to the issue of improving American immigration policy).

But I contend that it is also not wrong to insist that these and other contentious national issues are most appropriately solved by elected civilian officials and not by retired military officers. Interestingly, in July of 2020, General Mark Milley, the Chairman of the Joint Chiefs of Staff, was asked by US Representative Anthony Brown, D-MD, during a hearing held by the House Armed Services Committee, to ``comment on the naming of Army installations after Confederate soldiers,'' and whether the continued use of the names of these bases ``reflects the values'' that the military is trying to instill in the nation's armed forces \autocite{milley_mark_general_2020}? Milley's response was instructive. He offered the Congressmen the idea that even as the Department of Defense continues to take a hard look at the issue, the decision to remove those names should reflect the decision to adopt those names in the first place in that such a decision was ``a political decision'' \autocite{milley_mark_general_2020}. Milley's answer was the appropriate one, I contend, because it recognized that regardless of how he feels about this issue, the issue is not for him (or the military) to solve. Many other topics addressed by publications listed in Table 4, to include gun violence and immigration, fall into the same realm.

A sixth finding is that there are roughly three clusters of observations in Table Four with respect to the year of publication. In the first cluster (1998-1999), the publications reveal multiple grievances between certain retired military officers and the manner in which President Clinton was planning or using military force abroad. Only two of the five observations in this first cluster are coded as possibly having violated standards of military professionalism.

The second cluster refers to publications penned between 2004-2006. Most of these comprise part of or are related to the so-called ``Revolt of the Generals,'' which occurred near the 2004 US midterm elections and referred to sharp disagreements that entered the public arena between multiple US generals and admirals and President George W. Bush's Secretary of Defense, Donald Rumsfeld. These disagreements were directly related to these military leaders' disagreement about Rumsfeld's leadership regarding the planning and execution of the early years of the war in Iraq. Retired Army General Tommy Franks penned two observations that essentially provided public support for the President, George W. Bush, while Retired Army Major General Paul Eaton, a critic of Rumsfeld, authored two of the pieces published in this cluster. Of the seven observations published in Table 4 between 2004-2006, all but one contained possible violations of the concept of military professionalism.

The third and largest cluster occurs between 2016-2019. The 23 observations in this cluster reveals concern over the candidate and later President Donald Trump. The titles alone of many of these opinion pieces reveal, at a minimum, clear partisan preferences by many of the authors. Retired Air Force General Michael Hayden, Navy Admirals Mike Mullen and William McRaven, and Army General Stanley McChrystal are among the retired officers who authored several publications during this time period. All but four of the observations in this cluster are coded as having potentially violated standards of military professionalism. Of note, one observation in this cluster entitled, ``Fists Raised at West Point'' and published in the Washington Post is actually a compendium of two anonymous letters, each one taking an opposite stance on the appropriateness of an incident in which a photograph at West Point's 2016 graduation ceremony captured a graduating female cadet raising a clenched a fist, an action which some saw as an inappropriate political gesture related to the growing Black Lives Matter movement \autocite{unknown_fists_2016}.

\hypertarget{variation-in-military-professionalism-over-time}{%
\subsection{Variation in Military Professionalism Over Time}\label{variation-in-military-professionalism-over-time}}

Figure 3, below, plots both the total number and proportion of yearly opinion pieces authored by retired military officers that address topics of social policy affecting the military, domestic policy, and political administrations or figures. The points are the actual terms (number and proportion, respectively), while the blue line represents a smoothed line of best fit for the data. Both plots reveal that beginning with the first year of the data, 1979, there has a moderate increase until approximately 2005, followed by a decline until approximately 2011, followed by a sharp increase that has persisted until the present day. What explains these dynamics - both the increase that occurred between 1979 - 2005 as well as the sharper increase between 2012 - today - and the decline that occurred between roughly 2005 - 2011? According to the theory developed at the beginning of this paper, a significant reason stems from changes in political polarization and institutional credibility that have occurred over the past four decades.

\includegraphics{Tahkv1_files/figure-latex/figure 3-1.pdf}

To understand how measurements of political polarization have changed over time, I rely on the DW-NOMINATE (dynamic-weight, nominal three step estimation) scores that were originally developed by the political scientists Keith Poole and Howard Rosenthal in the early 1980s \autocite{jeffrey_b_lewis_voteview_2020}. The DW-NOMINATE data are one measure that uses roll-call voting data for each member of congress to assign a score to the member of congress on a liberal-conservative dimension, ranging from -1 (extreme liberal) to 1 (extreme conservative) \autocite{jeffrey_b_lewis_voteview_2020}. The data is frequently relied upon by academics, journalists, and the media to understand how a variety of political bodies, to include Congress and political parties, have changed over time. Over the time period examined in this paper, polarization in each chamber of Congress, the House of Representatives and the Senate, has increased \autocite{jeffrey_b_lewis_voteview_2020}. The distance in party mean DW-NOMINATE scores in the late 1970s was just less than .6, while the distance today for both chambers is above .8 \autocite{jeffrey_b_lewis_voteview_2020}.

Changes in the institutional credibility of the American military institution can be seen by examining public opinion data. Pollsters survey the American public annually to gauge their trust in a variety of institutions. Using Gallup opinion polling, one sees that since the late 1970s, the credibility of the American military as an institution has gradually increased \autocite{gallup_polls_confidence_2020}. In 1979, 54\% of those surveyed expressed either a great deal or quite a lot of confidence in the military, while in 2020, 72\% expressed the same \autocite{gallup_polls_confidence_2020}. These trends in political polarization and confidence in the military are displayed in Figure Four, below. The top and middle graph show changes in means between Republicans and Democrats in the House of Representatives (top graph) and Senate (middle graph) since 1979, while the bottom graph shows changes in the percentage of surveyed Americans who reported either a great deal of or a lot of confidence in the military institution over the same time period.

\includegraphics{Tahkv1_files/figure-latex/figure 4-1.pdf}

While it is possible that the trend in increasing institutional credibility of the US military over the past four decades may be at least partially responsible for the increasing trend in the number of opinion pieces published by retired military officers, a trend that the data supported earlier as shown in Figure 2, rising institutional credibility alone does not adequately explain the trends shown in Figure 3: that there has been a sharp increase over the past four decades in retired military officers penning pieces that address non-traditional military topics that lay outside the bounds of relevant military expertise, or pieces that express partisan sentiment. The theory laid out earlier in this paper predicted that as political polarization increases, three mechanisms are at work which have strong impacts on the military professional: it becomes increasingly difficult to remain politically neutral as partisan conflicts subsume a greater number of issues; as this occurs, strong feelings of dislike or disdain surface against those who hold on opposing view; and alterations in one's sense of loyalty arise, particularly as issues that individuals hold to most dearly raise the salience of one's identity.

To test if changes in political polarization adequately explain the changes in the qualitative nature (increasing trends of potential violations of military professionalism) of opinion pieces authored by retired military officers, I conduct a logit regression of the variable \emph{questionable\_professionalism} while holding institutional trust in the military constant at its mean (71 percent) and varying the value of the mean ideological distance between both major political parties in the House between 1979-2020. Conducting such a regression allows one to examine, given the data assembled, the predictive likelihood of publishing an opinion piece that violates military professionalism given varying levels of political polarization and constant institutional credibility in the military. These results are displayed in Figure 5, below.

\% Table created by stargazer v.5.2.2 by Marek Hlavac, Harvard University. E-mail: hlavac at fas.harvard.edu
\% Date and time: Thu, Dec 03, 2020 - 19:06:06
\% Requires LaTeX packages: dcolumn

\begin{table}[!htbp] \centering 
  \caption{Table 3: Logistic Regression Results} 
  \label{} 
\begin{tabular}{@{\extracolsep{5pt}}lD{.}{.}{-3} D{.}{.}{-3} D{.}{.}{-3} D{.}{.}{-3} } 
\\[-1.8ex]\hline 
\hline \\[-1.8ex] 
 & \multicolumn{4}{c}{\textit{Dependent variable:}} \\ 
\cline{2-5} 
\\[-1.8ex] & \multicolumn{1}{c}{leg\_crit\_part} & \multicolumn{1}{c}{crit\_part} & \multicolumn{1}{c}{leg\_crit\_part} & \multicolumn{1}{c}{crit\_part} \\ 
 & \multicolumn{1}{c}{Model 3} & \multicolumn{1}{c}{Model 4} & \multicolumn{1}{c}{Model 5} & \multicolumn{1}{c}{Model 6} \\ 
\\[-1.8ex] & \multicolumn{1}{c}{(1)} & \multicolumn{1}{c}{(2)} & \multicolumn{1}{c}{(3)} & \multicolumn{1}{c}{(4)}\\ 
\hline \\[-1.8ex] 
 Institutional Credibility & -0.057^{*} & -0.047 & -0.024 & -0.019 \\ 
  & (0.031) & (0.036) & (0.024) & (0.029) \\ 
  & & & & \\ 
 Congressional Polarization & 0.084^{***} & 0.088^{***} &  &  \\ 
  & (0.027) & (0.031) &  &  \\ 
  & & & & \\ 
 Affective Polarization &  &  & 0.033^{**} & 0.037^{**} \\ 
  &  &  & (0.013) & (0.015) \\ 
  & & & & \\ 
 Male & -0.708 & -0.856 & -0.848 & -0.976 \\ 
  & (1.256) & (1.254) & (1.262) & (1.263) \\ 
  & & & & \\ 
 Mixed Authorship & 1.340 & 1.265 & 1.324 & 1.253 \\ 
  & (1.755) & (1.739) & (1.759) & (1.746) \\ 
  & & & & \\ 
 Unknown Authorship & -16.178 & -15.996 & -16.366 & -16.206 \\ 
  & (787.186) & (798.975) & (789.107) & (799.686) \\ 
  & & & & \\ 
 Rank & 0.171^{**} & 0.022 & 0.167^{**} & 0.018 \\ 
  & (0.076) & (0.078) & (0.076) & (0.078) \\ 
  & & & & \\ 
 Constant & -4.977^{**} & -4.946^{**} & -2.567 & -2.256 \\ 
  & (2.155) & (2.477) & (1.945) & (2.173) \\ 
  & & & & \\ 
\hline \\[-1.8ex] 
Observations & \multicolumn{1}{c}{389} & \multicolumn{1}{c}{389} & \multicolumn{1}{c}{388} & \multicolumn{1}{c}{388} \\ 
Log Likelihood & \multicolumn{1}{c}{-180.151} & \multicolumn{1}{c}{-149.394} & \multicolumn{1}{c}{-182.477} & \multicolumn{1}{c}{-150.757} \\ 
Akaike Inf. Crit. & \multicolumn{1}{c}{374.301} & \multicolumn{1}{c}{312.787} & \multicolumn{1}{c}{378.954} & \multicolumn{1}{c}{315.514} \\ 
\hline 
\hline \\[-1.8ex] 
\textit{Note:}  & \multicolumn{4}{r}{$^{*}$p$<$0.1; $^{**}$p$<$0.05; $^{***}$p$<$0.01} \\ 
\end{tabular} 
\end{table}

The graph clearly shows a non-linear relationship between increasing political polarization and the likelihood that an opinion piece authored by a retired military officer will potentially violate aspects of military professionalism. While the confidence intervals at each point estimate are relatively large, the graph demonstrates that the predicted probability of violating military professionalism is different between relative extremes along the x axis. In other words, the political polarization levels that exist today, as measured by the ideological distance between members of the two prominent political parties in the house, is predicted to have resoundingly different impacts on military professionalism than the polarization levels that existed several decades ago. The polarization levels that existed several decades ago, in the late 1970s and 1980s, may have had negligible, if any impact, on the likelihood of a retired military officer violating military professionalism in the content of a published opinion piece, but the political landscape is simply different today. Contemporary polarization levels, insofar as the predictive power of the model is concerned, changed the environment: at contemporary polarization levels (the far right of the x axis), there is roughly a 17\% chance of a retired military officer's published opinion piece violating the standards and norms of military professionalism, with confidence intervals that range from roughly 11\% to nearly 25\%.

Given these dynamics, as well as the results of both Figures 3 and 5, ample evidence exists to confirm the second hypothesis, which stated that as a result of increasing political polarization, there will be an increase in recent years in the number of op-eds written about topics that are not relevant to military expertise, or that reveal inappropriate partisan sentiments, among authors which are a vital component of the institutional military, retired military officers.

\hypertarget{does-combat-attenuate-polarizations-impact-on-military-professionalism}{%
\subsection{Does Combat Attenuate Polarization's Impact on Military Professionalism?}\label{does-combat-attenuate-polarizations-impact-on-military-professionalism}}

Why did Figure 3 reveal a sharp drop between approximately 2005-2011 in both the real number and proportion of annual opinion pieces authored by retired military officers that address non-traditional military topics, such as social policy that impacts the military, domestic policy, and political administrations/figures? What was it about this time period that made it unlike any other span of years across the past four decades?

One compelling answer is that between these years, roughly 2005-2011, the military was heavily involved in combat operations to a point such that the institution was taxed tremendously, thereby necessitating that it remain focused on its wartime mission. Repeated deployments during the Global War on Terrorism, which included substantial surges of forces to both Iraq (2007-2008) and Afghanistan (2010-2011), as well as sustained US casualties resulting from these conflicts, possibly shaped the military institution so strongly such that even retired military officers acknowledged that the military's institutional energy required a laser-sharp focus on combat requirements. This is not to suggest the military was not heavily engaged in conflicts nor active around the glove at other times throughout the past four decades (which we know is not the case), but to instead suggest that even retired military officers are perhaps more sensitive to combat requirements and missions undertaken by the military as higher levels of continued operational strain and duress are endured. Though more data is required to fully test this argument, other scholars have argued similarly in that periods of transition, especially away from major combat operations and into eras of downsizing and relative peace, often seem to test the very essence of military professions in significant ways \autocite{snider_once_2012}.

\hypertarget{conclusion-will-the-military-remain-professional-in-a-polarized-setting}{%
\section{Conclusion: Will the Military Remain Professional in a Polarized Setting?}\label{conclusion-will-the-military-remain-professional-in-a-polarized-setting}}

The goal of this paper was to explore and examine the ways in which political polarization and institutional credibility have impacted the professionalism of the military over the past fifty years. Examining the published opinion pieces authored by retired military officers in major US newspapers revealed that although the vast number of total opinion publications uphold the military's norms and traditions concerning military professionalism, a couple of worrisome trends are evident in the data: first, retired military officers are increasingly writing about topics outside of the umbrella of traditional national security topics, and second, retired officers are expressing partisan preferences at a greater rate than they previously did. What do these trends mean, and should they cause concern? If so, what should be done about it?

First, these findings illuminate the power and pervasiveness of increasing political polarization, and in particular, its impact not only on political institutions, but also on professions, which the military prides itself on being. Thus, if the military, which has for decades adopted and promulgated norms of apolitical and nonpartisan expression, has been impacted by rising political polarization, it stands to reason that other political institutions and professions also have been impacted. While the specific manifestations of political polarization's increase and spread in non-military institutions and professions might be different, the fact that the military is experiencing a noticeable impact as a result of increasing political polarization speaks to its significant power and influence.

Indeed, evident in the content of opinion pieces authored by retired military officers over the past four decades are the three features of contemporary American political polarization developed in the theory proffered earlier in this paper: the loss of politically neutral space as more issues enter the partisan battleground, the creation of strong feelings of dislike for those who hold to opposing views on these issues, and perhaps worst for the military, alterations in an individual's concept of loyalty, which arises as the salience of one's identity is raised as the issues that an individual holds most dearly to are increasingly at stake. Moreover, the data supported what most scholars of American politics have argued for: a realization that polarizing politics across America's political landscape will likely not disappear anytime soon, regardless of which party holds the nation's highest office.

Second, these findings underscore the challenge that the military as a profession has in ``reigning in'' its retired officers and compelling them to stay out of partisan battles, and out of politics more broadly. Among the features and traits of a profession is that it is self-policing; however, the data indicates that current efforts by active military leaders to police and clean up the behavior of its retired elites is not working. Indeed, the letters and pleas of previous military leaders to retired military officers, including those written by now Retired Army General Martin Dempsey and Retired Marine Corps General Joseph Dunford, both of which were referenced in this paper, are simply not working. This suggests that stronger and more creative efforts are required.

Third, these findings demonstrate that the phrase ``politicization of the military,'' which has long been a proper concern of academics, journalists, the military, and citizens alike, should not be intellectually confined to notions of how politicians or other political leaders political leaders attempt to harness or exploit the military for political purposes or advantages. Though such efforts do occur and are a genuine cause for concern that continue to deserve both scholarly and practical attention, the evidence presented in this paper shows that ``politicization of the military'' is also occurring at the hands and through the words of particular retired military officers. Gaining a better understanding of the impacts of political polarization on the military profession remains a critical area for future research. Other scholars such as Michael Robinson have explored the specific ways that polarization is impacting the interactions between the military and the public \autocite{robinson_michael_danger_2018}. Though this paper focused on the ways that polarization may impact professional military behavior, it recognized but stopped short of discussing polarization's subsequent impact on various ``military outcomes,'' which may arguably include even more decisive events, such as achieving victory or defeat in battle, or the ability to formulate sound strategic advice. These outcomes and more, and the role of polarization in influencing them, deserve to be explored further.

Maintaining credibility and trust with the American public is crucial for the military, but should a polarized political climate endure, the military needs to also understand that credibility and trust can be abused to the point that they also can constitute significant risks. One such risk is that as polarization worsens, military leaders, to include military retirees, abuse the high level of credibility they currently enjoy with the American public to advocate for obviously partisan positions. Such actions, though, in a polarized setting, might actually result in an increase in the credibility of the military institution, at least among citizens who share the same partisan sentiments advocated for by military leaders, which might, somewhat perversely, entice more of the same behavior. A second risk, stemming from the first, is that a polarized setting sets the stage for the military institution to eventually openly jockey with civilian leaders, either for power or for control, a concern that generations of scholars have warned us about \autocites{finer_man_1962}{brooks_shaping_2008-1}. Were either of these risks to materialize, the character and nature of the American military as an institution, and likely, American democracy, would be irrevocably changed.

There are no simple solutions for the military with respect to the risks posed by rising political polarization. Two possible ideas, however, are apropos to this discussion. One involves identity, and the other, credibility. First, our nation's military is composed of soldiers who come from every walk of life throughout America, and accordingly, in the years to come, many future soldiers and leaders will bear the features of a polarized society. Thus, the military's challenge, and one which it must embrace itself for immediately, is integrating the multiple identities each soldier carries with them as people in such a way that each can still proudly wear the uniform and bear the identity of an American service member. Though this challenge is not new for the US military, increasing political polarization has significantly raised the level of difficulty the military faces in tackling such a vital challenge. Scholars who have studied multi-ethnic militaries across the world have argued that it is military leaders, not civilians, who ultimately devise effective solutions to the problems posed by soldiers who carry multiple loyalties, and interestingly, only as a result of combat requirements \autocite{peled_question_1998}. Perhaps the American challenge is even greater in that at least for the near term future, there is no major war that will necessitate the coming together of disparate loyalties and identities among future soldiers.

The second idea, and one that may sound like anathema to modern military leaders, is that perhaps the level of trust and credibility between the American public and the military institution should not be the ultimate indicator of the military's performance as a profession. In suggesting this, I am not suggesting that credibility and trust are not important or that the concept should be disregarded entirely. The American public is, after all, the client of the American military and ultimately, it is their opinion that counts.I am suggesting, however, that for the reasons laid out in this paper, credibility and trust are influenced by the mechanisms through which polarization works, to include the loss of politically-neutral space, increased negative feelings toward those who hold differing opinions, and most sinister of all, alterations in one's concept of loyalty. For these reasons, I am simply offering the notion that \emph{credibility can be deceiving} in polarized times. Now, more than ever before, military leaders will need to rely on individual moral courage over and above any other trait in order to perform their duties as military professionals.\\
\singlespace
\pagebreak

\printbibliography

\end{document}
